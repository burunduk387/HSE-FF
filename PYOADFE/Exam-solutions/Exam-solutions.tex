%
%Не забыть:
%--------------------------------------
%Вставить колонтитулы, поменять название на титульнике



%--------------------------------------

\documentclass[a4paper, 12pt]{article} 

%--------------------------------------
%Russian-specific packages
%--------------------------------------
%\usepackage[warn]{mathtext}
\usepackage[T2A]{fontenc}
\usepackage[utf8]{inputenc}
\usepackage[english,russian]{babel}
\usepackage[intlimits]{amsmath}
\usepackage{esint}
%--------------------------------------
%Hyphenation rules
%--------------------------------------
\usepackage{hyphenat}
\hyphenation{ма-те-ма-ти-ка вос-ста-нав-ли-вать}
%--------------------------------------
%Packages
%--------------------------------------
\usepackage{amsmath}
\usepackage{amssymb}
\usepackage{amsfonts}
\usepackage{amsthm}
\usepackage{latexsym}
\usepackage{mathtools}
\usepackage{etoolbox}%Булевые операторы
\usepackage{extsizes}%Выставление произвольного шрифта в \documentclass
\usepackage{geometry}%Разметка листа
\usepackage{indentfirst}
\usepackage{wrapfig}%Создание обтекаемых текстом объектов
\usepackage{fancyhdr}%Создание колонтитулов
\usepackage{setspace}%Настройка интерлиньяжа
\usepackage{lastpage}%Вывод номера последней страницы в документе, \lastpage
\usepackage{soul}%Изменение параметров начертания
\usepackage{hyperref}%Две строчки с настройкой гиперссылок внутри получаеммого
\usepackage[usenames,dvipsnames,svgnames,table,rgb]{xcolor}% pdf-документа
\usepackage{multicol}%Позволяет писать текст в несколько колонок
\usepackage{cite}%Работа с библиографией
\usepackage{subfigure}% Человеческая вставка нескольких картинок
\usepackage{tikz}%Рисование рисунков
\usetikzlibrary{circuits} % подключаем библиотеки, содержащие
\usetikzlibrary{circuits.ee} % УГО для схем
\usetikzlibrary{circuits.ee.IEC}
\usetikzlibrary{arrows} % подключаем библиотеки со стрелками
\usetikzlibrary{patterns} % и со штриховкой
\usepackage{float}% Возможность ставить H в положениях картинки
% Для картинок
\usepackage{misccorr}
\usepackage{lscape}
\usepackage{cmap}
\usepackage{bm}
\newtheorem{definition}{Опредление}



\usepackage{graphicx,xcolor}
\graphicspath{{Pictures/}}
\DeclareGraphicsExtensions{.pdf,.png,.jpg}

%----------------------------------------
%Список окружений
%----------------------------------------
\newenvironment {theor}[2]
{\smallskip \par \textbf{#1.} \textit{#2}  \par $\blacktriangleleft$}
{\flushright{$\blacktriangleright$} \medskip \par} %лемма/теорема с доказательством
\newenvironment {proofn}
{\par $\blacktriangleleft$}
{$\blacktriangleright$ \par} %доказательство
%----------------------------------------
%Список команд
%----------------------------------------
\newcommand{\grad}
{\mathop{\mathrm{grad}}\nolimits\,} %градиент

\newcommand{\diver}
{\mathop{\mathrm{div}}\nolimits\,} %дивергенция

\newcommand{\rot}
{\ensuremath{\mathrm{rot}}\,}

\newcommand{\Def}[1]
{\underline{\textbf{#1}}} %определение

\newcommand{\RN}[1]
{\MakeUppercase{\romannumeral #1}} %римские цифры

\newcommand {\theornp}[2]
{\textbf{#1.} \textit{ #2} \par} %Написание леммы/теоремы без доказательства

\newcommand{\qrq}
{\ensuremath{\quad \Rightarrow \quad}} %Человеческий знак следствия

\newcommand{\const}{\text{const}} % Написание const в формулах

\newcommand{\qlrq}
{\ensuremath{\quad \Leftrightarrow \quad}} %Человеческий знак равносильности

\renewcommand{\phi}{\varphi} %Нормальный знак фи

\renewcommand{\epsilon}{\varepsilon}

\newcommand{\me}
{\ensuremath{\mathbb{E}}}

\newcommand{\md}
{\ensuremath{\mathbb{D}}}



%\renewcommand{\vec}{\overline}




%----------------------------------------
%Разметка листа
%----------------------------------------
\geometry{top = 3cm}
\geometry{bottom = 2cm}
\geometry{left = 1.5cm}
\geometry{right = 1.5cm}
%----------------------------------------
%Колонтитулы
%----------------------------------------
\pagestyle{fancy}%Создание колонтитулов
\fancyhead{}
%\fancyfoot{}
\fancyhead[R]{\textsc{Билеты к экзамену}}%Вставить колонтитул сюда
%----------------------------------------
%Интерлиньяж (расстояния между строчками)
%----------------------------------------
%\onehalfspacing -- интерлиньяж 1.5
%\doublespacing -- интерлиньяж 2
%----------------------------------------
%Настройка гиперссылок
%----------------------------------------
\hypersetup{				% Гиперссылки
	unicode=true,           % русские буквы в раздела PDF
	pdftitle={Заголовок},   % Заголовок
	pdfauthor={Автор},      % Автор
	pdfsubject={Тема},      % Тема
	pdfcreator={Создатель}, % Создатель
	pdfproducer={Производитель}, % Производитель
	pdfkeywords={keyword1} {key2} {key3}, % Ключевые слова
	colorlinks=false,       	% false: ссылки в рамках; true: цветные ссылки
	linkcolor=blue,          % внутренние ссылки
	citecolor=blue,        % на библиографию
	filecolor=magenta,      % на файлы
	urlcolor=blue           % на URL
}
%----------------------------------------
%Работа с библиографией 
%----------------------------------------
\renewcommand{\refname}{Список литературы}%Изменение названия списка литературы для article
%\renewcommand{\bibname}{Список литературы}%Изменение названия списка литературы для book и report
%----------------------------------------
\begin{document}
	\begin{titlepage}
		\begin{center}
			$$$$
			$$$$
			$$$$
			$$$$
			{\Large{НАЦИОНАЛЬНЫЙ ИССЛЕДОВАТЕЛЬСКИЙ УНИВЕРСИТЕТ}}\\
			\vspace{0.1cm}
			{\Large{ВЫСШАЯ ШКОЛА ЭКОНОМИКИ}}\\
			\vspace{0.25cm}
			{\large{Факультет физики}}\\
			\vspace{5.5cm}
			{\Huge\textbf{{ОАД}}}\\%Общее название
			\vspace{1cm}
			{\LARGE{<<Билеты к экзамену>>}}\\%Точное название
			\vspace{1cm}
			{\LARGE{Лектор: Корнилов М. В.}}\\%Лектор
			\vspace{2cm}
			\vfill
			\includegraphics[width = 0.2\textwidth]{HSElogo}\\
			\vfill
			Москва\\
			2021
		\end{center}
	\end{titlepage}
	
	\tableofcontents
	\newpage
	\addcontentsline{toc}{section}{Замечания и благодарности}
	\section*{Замечания и благодарности}
	Данный конспект написан студентами и для студентов. Он может содержать опечатки, неточности и серьёзные смысловые ошибки.
	Над ним работали:
	\begin{itemize}
		\item Написание:
		\subitem М. Блуменау
	\end{itemize}
	\href{}{Ссылка на записи лекций} 
	\newline
	\href{}{Ссылка на репозиторий с исходными файлами} 
	\newpage
	\addcontentsline{toc}{section}{Структуры данных}
	\section*{Структуры данных}
	\addcontentsline{toc}{subsection}{Билет 1-3. O-нотация. Приведите примеры алгоритмов над структурами данных с константным и квадратичным (с линейным и логарифмическим) по числу элементов временем работы. Приведите примеры, в которых выбор асимптотически более медленного с точки зрения O-нотации алгоритма предпочтителен.}
	\subsection*{Билет 1-3. O-нотация. Приведите примеры алгоритмов над структурами данных с константным и квадратичным (с линейным и логарифмическим) по числу элементов временем работы. Приведите примеры, в которых выбор асимптотически более медленного с точки зрения O-нотации алгоритма предпочтителен.}
	В информатике временная сложность алгоритма определяется как функция от длины строки, представляющей входные данные, равная времени работы алгоритма на данном входе. Временная сложность алгоритма обычно выражается с использованием нотации «O» большое, которая учитывает только слагаемое самого высокого порядка, а также не учитывает константные множители, то есть коэффициенты. Если сложность выражена таким способом, говорят об асимптотическом описании временной сложности, то есть при стремлении размера входа к бесконечности. Временная сложность обычно оценивается путём подсчёта числа элементарных операций, осуществляемых алгоритмом. Время исполнения одной такой операции при этом берётся константой, то есть асимптотически оценивается как $O(1)$. В таких обозначениях полное время исполнения и число элементарных операций, выполненных алгоритмом, отличаются максимум на постоянный множитель, который не учитывается в O-нотации.
	
	Примеры алгоритмов:
	\begin{itemize}
		\item $O(1)$, константный:
		\subitem Определение чётности целого числа (представленного в двоичном виде)
		\item $O(n^{2})$, квадратичный:
		\subitem Сортировка пузырьком (попарное сравнение соседних элементов)
		\subitem Сортировка вставками (рассматриваем по элементы по одному, каждый новый элемент - в подходящее место)
		\item $O(n)$, линейный:
		\subitem Поиск наименьшего или наибольшего элемента в неотсортированном массиве
		\item $O(log(n))$, логарифмический:
		\subitem Бинарный поиск (метод деления пополам)
	\end{itemize}
	
	Пример, когда асимптотически более медленный алгоритм предпочтителен:
	
	Сложность вычисления произведения матриц по определению составляет $O(n^{3})$, однако существуют более эффективные алгоритмы, применяющиеся для больших матриц.
	
	Первый алгоритм быстрого умножения больших матриц был разработан Фолькером Штрассеном в 1969. В основе алгоритма лежит рекурсивное разбиение матриц на блоки $2 \cdot 2$. Штрассен доказал, что такие матрицы можно некоммутативно перемножить с помощью семи умножений, поэтому на каждом этапе рекурсии выполняется семь умножений вместо восьми. В результате асимптотическая сложность этого алгоритма составляет $O(n^{\log _{2}7})\approx O(n^{2.81})$. Недостатком данного метода является бОльшая сложность программирования по сравнению со стандартным алгоритмом, слабая численная устойчивость и бОльший объём используемой памяти. Разработан ряд алгоритмов на основе метода Штрассена, которые улучшают численную устойчивость, скорость по константе и другие его характеристики. Тем не менее, в силу простоты алгоритм Штрассена остаётся одним из практических алгоритмов умножения больших матриц.
	
	В дальнейшем оценки скорости умножения больших матриц многократно улучшались. Однако эти алгоритмы носили теоретический, в основном приближённый характер. В силу неустойчивости алгоритмов приближённого умножения в настоящее время они не используются на практике. Лучший на данный момент - $O(n^{2.37})$.
	
	Таким образом, можно сказать, что такие примеры в основном связаны с используемой памятью и устойчивостью.
	\addcontentsline{toc}{subsection}{Билет 4.  Сортировка. Приведите примеры методов сортировки.}
	\subsection*{Билет 4.  Сортировка. Приведите примеры методов сортировки.}
	
	\section*{Оптимизация}
	\addcontentsline{toc}{subsection}{Билет 1. Задачи оптимизации. Их классификация. Примеры}
	\subsection*{Билет 1. Задачи оптимизации. Их классификация. Примеры}
	
\end{document}