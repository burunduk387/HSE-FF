\documentclass[prb,papersize=a4paper,notitlepage]{revtex4-1}%
\usepackage{hyperref}
\usepackage{enumitem}
\usepackage{nicefrac}
\usepackage{amsmath}
\usepackage{graphicx}
\usepackage{amsfonts}
\usepackage{physics}
\usepackage{amssymb}
\usepackage{bm}
\usepackage[utf8]{inputenc}
\usepackage[russian]{babel}
\usepackage{listings}


\begin{document}

\title{Вычислительная физика, Осень 2020 ВШЭ. Задание 1.\footnote{Дополнительно указаны: (количество баллов за задачу)[имя задачи на nbgrader]}}
\maketitle


\begin{enumerate}
\item \textbf{(15)} Вспомните \href{https://github.com/ev-br/CP2020/blob/master/week_0_recurrence.ipynb}{разобранный на семинаре пример}. Теперь рассмотрите интеграл $I_n(\alpha) = \int_0^1 \frac{x^n}{x+\alpha} dx$ и получите i) рекуррентное соотношение, связывающее $I_{n}$ c $I_{n-1}$ и ii) явное выражение для $I_0(\alpha)$. Вычислите (прямой и обратной) рекурсией значения $I_{25}(0.1)$ и $I_{25}(10)$. Обсудите результат.
\item \textbf{(10) [Quadratic]} \href{https://github.com/ev-br/CP2020/blob/master/week_1_quadratic.ipynb}{Реализуйте функцию, возвращающую пару решений квадратного уравнения} (следуйте инструкциям по ссылке).
\item \textbf{(10)} Рассмотрите следующую функцию:
\lstset{language=Python}
\lstset{frame=lines}
\lstset{label={lst:code_direct}}
\lstset{basicstyle=\ttfamily}
\begin{lstlisting}
def recur(n):
    if n == 0:
        return 1
    if n == 1:
        return -3
    return -recur(n-1) + 6*recur(n-2)
\end{lstlisting}

Чему будет равен результат вызова `recur(2020)'? Диапазон определения целых чисел считать неограниченным (т.е., целые числа не переполняются), размер стека также считать неограниченным (т.е. максимальное число рекурсивных вызовов не ограничено).
\item \textbf{(5)} Рассмотрите (считая $\delta>0$) матрицу $A=\begin{pmatrix}
1 & 10 \\
\delta &1 
\end{pmatrix}$. Пусть $\epsilon(\delta)$ -- наибольшее собственное значение $A$. Найдите число обусловленности этого собственного значения $\kappa(\delta)=\frac{d\epsilon(\delta)}{d\delta}$ для $\delta=10$ и $\delta=0.1$.

\item \textbf{(7)} Убедитесь, что функция
\lstset{language=Python}
\lstset{frame=lines}
\lstset{label={lst:code_direct}}
\lstset{basicstyle=\ttfamily}
\begin{lstlisting}
import math
def round_to_n(x, n): 
    if x == 0:
        return x
    else:
        return round(x, -int(math.floor(math.log10(abs(x)))) + (n - 1))
\end{lstlisting}
округляет $x$ до $n$ значащих цифр.
Программа, вычисляющая $\sum_{k=1}^{3000}k^{-2}\approx 1.6446$ последовательным суммированием членов ряда с округлением промежуточных результатов до 4х знаков выглядит следующим образом:
\lstset{language=Python}
\lstset{frame=lines}
\lstset{label={lst:code_direct}}
\lstset{basicstyle=\ttfamily}
\begin{lstlisting}
res = 0
for k in range(1,3001):
    res = round_to_n(res+1/k**2, 4)
\end{lstlisting}
Несмотря на отсутствие вычитаний (и связанных с ними сокращений), такой код позволяет получить только две значащие цифры точного ответа. Объясните, почему так происходит и предложите более удачный способ вычисления этой суммы (ограничиваясь 4мя значащими цифрами для промежуточных результатов).
\end{enumerate}

\end{document}